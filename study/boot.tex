% Linux 3.1 rc5 부팅과정 분석
\documentclass[a4paper,11pt]{report}

\usepackage[T1]{CJK,hyperref,amsmath,graphicx,multirow,listings,color,xcolor,caption,fontenc,tikz}
\usetikzlibrary{shapes,arrows}

% \DeclareCaptionFont{white}{\color{white}}
% \DeclareCaptionFormat{listing}{\colorbox{gray}{\parbox{\textwidth}{#1#2#3}}}
% \captionsetup[lstlisting]{format=listing,labelfont=white,textfont=white  ,breaklines=true}
\lstset{language=C,captionpos=b,tabsize=3,frame=lines,keywordstyle=\color{blue},commentstyle=\color{darkgreen},stringstyle=\color{red},numbers=left,numberstyle=\tiny,numbersep=5pt,breaklines=true,showstringspaces=false,basicstyle=\footnotesize,emph={label}}

% 쪽번호, index 생성, figure-float, item, hyperref

% 그림 첨부는 includegraphics 사용, 다만 스케일 조정이 필요. eps사용이 좋음
%\includegraphics{head64.eps}
%\includegraphics{init.eps}


\begin{document}
\begin{CJK}{UTF8}{}

\title{Linux x86\_64 부팅과정 분석\\
Version: 3.2\\}
\author{
%\and{\email{} x86-8}
\and{iamroot x86 8차 초급}
}
\date{}


\CJKfamily{mj}
\maketitle

\chapter{Boot loader}
PC의 전원을 켜면 BIOS가 POST(Power-on self-test)과정을 완료하고 부팅할 디바이스의 MBR(Master boot record)을 메모리의 0x7C00번지로 읽어들인다.
%CPU 초기화 f000:fff0 확인요망
\footnote{EFI는 GPT로 부팅하며 MBR를 사용하는 전통적인 BIOS와 다르다.}

\section{Boot protocol}
부트로더는 커널로 제어권을 넘길때 boot protocol을 통해서 리눅스 커널로 정보를 전달한다.\footnote{Documentation/x86/boot.txt}\\
header 정보는 header.S\footnote{arch/x86/boot/header.S}에 있으며 커널 이미지(vmlinuz)의 첫 2섹터(BOOTSECT, SETUP)에 걸쳐있다. \footnote{setup.ld와 header.S의 .header 섹션 참조}\\
부트 프로토콜은 setup의 섹터수, 커널이미지 위치, command line, initrd 주소등 중요한 정보를 가지고 있다.

\section{EBDA}
EBDA(extended BIOS data area)는 0x4xx의 BDA(BIOS data area)영역에서 확장된 BIOS 정보를 가지고 있다. 기본메모리(conventional memory:640KB)의 경계인 0xA0000 아래에 위치한다.
protocol 버전 2.02부터 가변적인 크기의 EBDA를 지원하기 때문에 lilo는 EBDA가 일정크기를 초과하면\footnote{int 12h의 크기로 확인한다.} 0x90000 아래에 setup코드를 위치시킨다.

\section{bzImage}
커널을 make하면 압축된 커널이 나오는데 bzImage는 최종적인 결과물이다.
% bzImage에 관한 추가 정보 필요!
lilo는 압축된 커널이 bzImage일때 두 부분으로 나눠서 읽어들인다.\\
보통 bzImage는 다음과 같이 구성되어 있다.\footnote{사실은 더 복잡하다. arch/x86/boot/compressed/Makefile 참조}\\\\
% table 표그리기
\begin{tabular}{|c|c|c|c|}
\hline
& 내용 & 섹터크기 & 메모리 주소\\ \hline
BOOTSECT & floppy 부팅\footnotemark[7] + header & 1 & 0x90000 \\ \hline
SETUP & header + 초기화 보호모드 진입 & 4\sim63 & 0x90200 \\ \hline % 4~63이 맞는지 확인요망
startup\_32 & 압축 해제 & & 0x100000 \\ \hline
piggy.o & 압축된 커널 본체 & & 0x1000000\footnotemark[8] \\ \hline
\end{tabular}
\footnotetext[7]{2.6이상에서 BOOTSECT를 통한 floppy booting은 지원되지 않는다}
\footnotetext[8]{압축된 상태에서는 .rodata..compressed 섹션에 위치한다. 압축이 풀리면 0x1000000 16M에 위치한다.}

\\
이미지 생성은 setup는 arch/x86/boot/setup.ld에서 setup를 만들고 compressed와 boot 두 디렉토리의 vmlinux.lds.S에서 나온 setup.bin과 vmlinux.bin을 arch/x86/boot/tools/build.c가 하나의 이미지로 합친다.

\chapter{SETUP}
커널의 첫 섹터인 bootsect는 2.6이전에 플로피디스크로 부팅하는 코드다. 이제 이 기능은 지원하지 않기 때문에 bootsect로 진입하면 에러메세지를 출력한다. 
bootsect의 일부와 setup의 첫부분에는 커널의 옵션들이 담긴 header가 있어서 부트로더과 커널은 이 부분의 정보를 참조한다.\\\\
 부트로더는 커널을 로드하는 작업을 마치면 setup으로 제어권을 넘긴다. MBR로 부팅하면 16비트(real mode)로 시작되기 때문에 확장된 기능들을 사용하려면 순차적으로 32비트(protected mode)와 64비트(long mode)로 넘어가야 한다. setup은 본 커널로 들어가기 전에 일부 초기화를 하고 보호모드로 전환한다.

\section{header} %arch/x86/boot/header.S
header 정보는 첫번째 섹터와 두번째 섹터에 걸쳐있다. \footnote{arch/x86/boot/header.S}\\
\begin{lstlisting}[label=lst:header1,caption=arch/x86/boot/header.S]
	.section ".header", "a"
	.globl	hdr
hdr:
setup_sects:	.byte 0			/* Filled in by build.c */
root_flags:	.word ROOT_RDONLY
syssize:	.long 0			/* Filled in by build.c */
ram_size:	.word 0			/* Obsolete */
\end{lstlisting}

\autoref{lst:header1}은 header의 시작부분이다. \autoref{lst:setup.ld}로 497부터 header가 위치하는걸 알 수 있다.

\begin{lstlisting}[label=lst:setup.ld,caption=arch/x86/boot/setup.ld]
	. = 0;
	.bstext		: { *(.bstext) }
	.bsdata		: { *(.bsdata) }

	. = 497;
	.header		: { *(.header) }
\end{lstlisting}


두번째 섹터(setup)에는 start\_of\_setup로의 점프코드가 있고 그 뒤에 나머지 header가 있다.\\
start\_of\_setup에서는 ds와 ss가 다르면 스택재계산을 하고 bss를 초기화하고 main을 호출한다.\footnote{돌아오지 않는 call이다. ret하면 ``No setup signature found..''를 출력하고 멈춘다.}

\section{arch/x86/boot/main.c}
아래는 main.c에 있는 함수들 목록과 기능 목록들이다.\\

\begin{tabular}{|c|c|} \hline
함수명 & 기능 \\\hline
copy\_boot\_params() & boot\_params.hdr로 header 복사 \\\hline
console\_init() & earlyprintk 옵션이 확인 \& 시리얼 포트 초기화 \\\hline
init\_heap() & header 참조해서 heap\_end 설정\\\hline
validate\_cpu()	 & CPU 레벨(386,486,...,64비트) 체크\\\hline
set\_bios\_mode() & BIOS에 64비트 사용을 알린다. \\\hline
detect\_memory() & e820, e801, 88 메모리 정보 수집\\\hline
keyboard\_set\_repeat() & 키입력 대기/반복 주기를 최대로 설정 \\\hline
query\_mca() & BIOS를 통해 MCA 정보 수집 \\\hline
query\_ist() & Intel speed step 정보 수집 \\\hline
query\_apm\_bios() & APM 정보 수집 (요즘은 ACPI 사용)\\\hline
query\_edd() & EDD 옵션 확인 및 EDD 정보 수집 \\\hline
set\_video() & 기본 비디오 모드 탐색과 설정\\\hline
go\_to\_protected\_mode() & a20, co-processor, idt, gdt 초기화\\
& 인터럽트 금지, 보호모드 진입\\
\hline

\end{tabular}

\section{arch/x86/boot/pm.c, arch/x86/boot/pmjump.S}
pm.c은 간단한 초기화 후에 pretected\_mode\_jump함수를 호출한다. 이 때 커널이 점프할 주소\footnote{boot\_params.hdr.code32\_start : 보통 1M에 위치한다.}를 인자로 넘겨준다. 1M이상의 메모리는 16비트 모드에서 다룰수 없기 때문에 32비트로 전환한뒤 커널 본체가 있는 코드로 점프하게 될 것이다.

PE비트를 켜서 보호모드로 들어가면 LDT와 TR을 더미 값으로 초기화한다.\footnote{intel vt이 테이블이 완전히 초기화 된 상황을 좋아하기 때문이다.}
이제 setup는 끝나고 본격적인 커널로 들어간다.

\chapter{COMPRESSED}
커널은 보통 압축되어 있다. 먼저 압축을 풀어서 원하는 위치(보통 16M)에 위치시키는 코드가 필요하다.
이 부분의 코드는 compress 디렉토리에 있다.
\section{arch/x86/boot/compressed/head\_64.S}
먼저 64비트 코드를 실행할 수 있는지 cpu 확인 후 커널을 위치시킬 곳을 계산한다. RELOCATABLE 옵션이 켜있으면 동적으로 계산하고 아니면 위치를 고정한다. 보통 압축푼 커널은 16M에 위치한다.\\
head\_64는 startup\_32와 startup\_64가 있다.

\subsection{startup\_32}
\begin{enumerate}
\item 임시로 gdt를 설정한다.
\item PAE를 켠다. long mode로 가려면 PAE와 paging을 켜야 한다.
\item 페이지 6개(24KB)를 0으로 초기화
\item LV4 1, LV3 1, LV2 4페이지를 할당해 4GB의 가상메모리의 공간을 물리주소에 1:1로 2M단위 페이징한다.
\item MSR의 EFER비트를 켜서 64비트(long mode)로 진입한다.
\item 스택에 주소를 넣고 long return하는 방법으로 startup\_64로 점프한다.
\end{enumerate}

\subsection{startup\_64}
\begin{enumerate}
\item 세그먼트, ldt, tss 재설정
\item 64비트로 압축풀 위치를 재계산한다.\\
여기서 사용되는 z\_extract\_offset 값은 mkpiggy.c로 생성된 piggy.S로 구해지는데 압축풀린크기-압축된크기에 약간의 공간을 더해준 것이다.
\item 1M에 위치한 현재 코드를 16M부터 압축풀릴 길이만큼 뒤에 복사한다.
\item 복사한 코드로 점프
\item BSS 초기화
\item 압축해제
\item 섹션을 LMA로 재배치
\item 압축푼 커널위치로 점프
\end{enumerate}

\chapter{KERNEL}
드디어 진짜 커널로 들어왔다.
어셈블리 코드로 몇가지 초기화 작업을 해준 뒤에 c 코드로 넘어간다.
\section{arch/x86/kernel/head\_64.S}
작업 순서는 다음과 같다.
\begin{enumerate}
\item 실행중인 코드시작위치와 컴파일 위치의 차이를 구해 페이징 계산에 더해준다.
\item 전체적인 임시 paging map을 만든다.
\item secondary\_startup\_64로 점프한다. 나중에 trampoline에서 이쪽을 사용한다.
\item PAE와 PGE를 켜고 cr3에 페이징 맵을 재설정한다.
\item SYSCALL과 No Execute 관련 비트를 EFER에서 설정
\item early\_gdt\_descr로 gdt 설정 & 세그먼트 초기화
\item x86\_64\_start\_kernel로 점프
\end{enumerate}

여기서 스택을 재설정 하는데 나중에 current\_thread\_info 매크로로 이 커널 스택의 thread 정보를 얻을수 있다.
8KB로 mask한 부분이 시작점이 되며 thread_info 구조체가 있다.

\section{arch/x86/kernel/head64.c}
x86\_64\_start\_kernel는 head64.c에 있다.
\section{init/main.c}



% Define block styles
\tikzstyle{decision} = [diamond, draw, fill=blue!20, 
    text width=4.5em, text badly centered, node distance=3cm, inner sep=0pt]
\tikzstyle{block} = [rectangle, draw, fill=blue!20, 
    text width=5em, text centered, rounded corners, minimum height=4em]
\tikzstyle{line} = [draw, -latex']
\tikzstyle{cloud} = [draw, ellipse,fill=red!20, node distance=3cm,
    minimum height=2em]

\begin{tikzpicture}[node distance = 2cm, auto]
    % Place nodes
    \node [block] (init) {initialize model};
    \node [cloud, left of=init] (expert) {expert};
    \node [cloud, right of=init] (system) {system};
    \node [block, below of=init] (identify) {identify candidate models};
    \node [block, below of=identify] (evaluate) {evaluate candidate models};
    \node [block, left of=evaluate, node distance=3cm] (update) {update model};
    \node [decision, below of=evaluate] (decide) {is best candidate better?};
    \node [block, below of=decide, node distance=3cm] (stop) {stop};
    % Draw edges
    \path [line] (init) -- (identify);
    \path [line] (identify) -- (evaluate);
    \path [line] (evaluate) -- (decide);
    \path [line] (decide) -| node [near start] {yes} (update);
    \path [line] (update) |- (identify);
    \path [line] (decide) -- node {no}(stop);
    \path [line,dashed] (expert) -- (init);
    \path [line,dashed] (system) -- (init);
    \path [line,dashed] (system) |- (evaluate);
\end{tikzpicture}


%\includegraphics{cpu_dev_struct.jpg}




\subsection{.ref section}
.ref 섹션. -> init으로 연결?

\subsection{init\_memory\_mapping}
특정 영역의 메모리를 초기화한다.
find\_early\_table\_space로 메모리 블럭을 다섯개의 배열로 나눈뒤 합치고 kernel\_physical\_mapping\_init 함수가 PGD부터 테이블을 생성한다.
커널이 쓰는 페이징 시작주소(cr3)는 swapper\_pg\_dir 로 연결되어 있으며 이전에 초기화한 테이블을 재사용 하기도 한다.


\end{CJK}
\end{document}

